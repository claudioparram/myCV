% Use the "normalphoto" option if you want a normal photo instead of cropped to a circle
% \documentclass[10pt,a4paper,normalphoto]{altacv}

\documentclass[10pt,letterpaper,withhyper]{altacv}
%% AltaCV uses the fontawesome5 and packages.
%% See http://texdoc.net/pkg/fontawesome5 for full list of symbols.

% Change the page layout if you need to
\geometry{left=1.25cm,right=1.25cm,top=1.5cm,bottom=1.5cm,columnsep=1.0cm}

% The paracol package lets you typeset columns of text in parallel
\usepackage{paracol}

% Change the font if you want to, depending on whether
% you're using pdflatex or xelatex/lualatex
% WHEN COMPILING WITH XELATEX PLEASE USE
% xelatex -shell-escape -output-driver="xdvipdfmx -z 0" sample.tex
\ifxetexorluatex
  % If using xelatex or lualatex:
  \setmainfont{Roboto Slab}
  \setsansfont{Lato}
  \renewcommand{\familydefault}{\sfdefault}
\else
  % If not using pdflatex:
  \usepackage[rm]{roboto}
  \usepackage[defaultsans]{lato}
  \usepackage{sourcesanspro}
  \renewcommand{\familydefault}{\sfdefault}
\fi

% Change the colours if you want to
\definecolor{SlateGrey}{HTML}{2E2E2E}%1
\definecolor{DarkPastelRed}{HTML}{450808}%2
\definecolor{PastelRed}{HTML}{8F0D0D}%3
\definecolor{GoldenEarth}{HTML}{E7D192}%4
\definecolor{color1}{HTML}{003545}
\definecolor{color2}{HTML}{00454A}
\definecolor{color3}{HTML}{3C6562}
\definecolor{color4}{HTML}{ED6363}
\colorlet{name}{black}
\colorlet{tagline}{color3}
\colorlet{heading}{color2}
\colorlet{headingrule}{color4}
\colorlet{subheading}{color3}
\colorlet{accent}{color3}
\colorlet{emphasis}{color1}
\colorlet{body}{black}


% Change some fonts, if necessary
\renewcommand{\namefont}{\Huge\rmfamily\bfseries}
\renewcommand{\personalinfofont}{\footnotesize}
\renewcommand{\cvsectionfont}{\LARGE\rmfamily\bfseries}
\renewcommand{\cvsubsectionfont}{\large\bfseries}


% Change the bullets for itemize and rating marker
% for \cvskill if you want to
\renewcommand{\cvItemMarker}{{\small\textbullet}}
\renewcommand{\cvRatingMarker}{\faCircle}
% ...and the markers for the date/location for \cvevent
\renewcommand{\cvDateMarker}{\faCalendar*[regular]}
%\renewcommand{\cvLocationMarker}{\faMapMarker}

%{title}{company or institution}{time period}{location}%
% If your CV/résumé is in a language other than English,
% then you probably want to change these so that when you
% copy-paste from the PDF or run pdftotext, the location
% and date marker icons for \cvevent will paste as correct
% translations. For example Spanish:
\renewcommand{\locationname}{Ubicación}
\renewcommand{\datename}{Fecha}


%% Use (and optionally edit if necessary) this .tex if you
%% want to use an author-year reference style like APA(6)
%% for your publication list
% \input{pubs-authoryear.tex}

%% Use (and optionally edit if necessary) this .tex if you
%% want an originally numerical reference style like IEEE
%% for your publication list
\input{pubs-num.tex}

%% sample.bib contains your publications
\addbibresource{my_pubs.bib}

\begin{document}
\name{Claudio A. Parra M.}
\tagline{Desarrollador Python | Ing. Civil en Telecomunicaciones}
%% You can add multiple photos on the left or right
% \photoR{2.8cm}{alta_cv/Globe_High}
% \photoL{2.5cm}{Yacht_High,Suitcase_High}
\headline{\input{sections/headline.tex}}
\personalinfo{%
  % Not all of these are required!
  \email{clmaldonadop@gmail.com}
  \location{Castro, Región de Los Lagos, Chile}
  % \homepage{www.homepage.com}
  \linkedin{claudiopm}
  \github{claudioparram}
  %\orcid{0000-0000-0000-0000}
  %% You can add your own arbitrary detail with
  %% \printinfo{symbol}{detail}[optional hyperlink prefix]
  % \printinfo{\faPaw}{Hey ho!}[https://example.com/]
  \NewInfoField{wsp}{\faWhatsapp}[https://wa.me/]
  \wsp{+56962034487}
  %% Or you can declare your own field with
  %% \NewInfoFiled{fieldname}{symbol}[optional hyperlink prefix] and use it:
  % \NewInfoField{gitlab}{\faGitlab}[https://gitlab.com/]
  % \gitlab{your_id}
  %%
  %% For services and platforms like Mastodon where there isn't a
  %% straightforward relation between the user ID/nickname and the hyperlink,
  %% you can use \printinfo directly e.g.
  % \printinfo{\faMastodon}{@username@instace}[https://instance.url/@username]
  %% But if you absolutely want to create new dedicated info fields for
  %% such platforms, then use \NewInfoField* with a star:
  % \NewInfoField*{mastodon}{\faMastodon}
  %% then you can use \mastodon, with TWO arguments where the 2nd argument is
  %% the full hyperlink.
  % \mastodon{@username@instance}{https://instance.url/@username}
}

\makecvheader
%% Depending on your tastes, you may want to make fonts of itemize environments slightly smaller
\AtBeginEnvironment{itemize}{\small}

%% Set the left/right column width ratio to 6:4.
\columnratio{0.65}

% Start a 2-column paracol. Both the left and right columns will automatically
% break across pages if things get too long.
\begin{paracol}{2}

\cvsection{\faLaptop Cursos y Certificaciones}
% Awesome Source CV LaTeX Template
%
% This template has been downloaded from:
% https://github.com/darwiin/awesome-neue-latex-cv
%
% Author:
% Christophe Roger
%
% Template license:
% CC BY-SA 4.0 (https://creativecommons.org/licenses/by-sa/4.0/)

%Section: Scholarships and additional info
\cvcert{2024}{Google AI Essentials \url{https://coursera.org/verify/4PX6S8XR3RDE}}{Google - Coursera}
\cvcert{2024}{EF SET English Certificate C1 Advanced (66/100)}{EF Standard English Test}
\cvcert{2020}{Tools for Database Modeling and SQL Queries.}{Pontificia Universidad Católica de Chile}
\cvcert{2020}{Advanced Python programming tools for data processing.}{Pontificia Universidad Católica de Chile}
\cvcert{2019}{Python programming tools for data processing.}{Pontificia Universidad Católica de Chile}
\cvcert{2019}{SOAR Engineer, Security Analyst, Administrator.}{Demisto}
\cvcert{2016}{Accredited Configuration Engineer.}{Palo Alto Networks}


\cvsection{\faSuitcase Experiencia Profesional}
%%%%%
% \cvevent{Job Title 1}{Company 1}{Month 20XX -- Ongoing}{Location}
% \begin{itemize}
% \item Job description 1
% \item Job description 2
% \end{itemize}

% \divider
%%%%%
\cvevent{Software Developer III - Contractor}{UST Global}{November 2024 - Today}{Santiago (Remote)}{logos/UST_logo.jpg}
\begin{itemize}
  \item Senior python developer in a big financial company.
  \item Contributed to cost optimization features for cloud infrastructure.
  \item \faStar Built a cost alerts system for company's gcp projects.
  \item Built infrastructure for Data Management in GCP.
\end{itemize}
\techenv{Python, Fast API, Git, VS Code, Linux, GCP, Airflow, Jenkins, Terraform}

\divider

\cvevent{Python Developer - Contractor}{Tata Consultancy Services}{January 2022 - March 2024}{Santiago (Remote)}{logos/tcs-logo.png}
\begin{itemize}
  \item Mid level python developer in a big financial company.
  \item Participated in the lifecycle of one of the client's Tool.
  \item The tool is an Agile project, with scrum and the \faSpotify Spotify model.
  \item \faStar Customized the Operator and Hook for Jenkins in Airflow. 
\end{itemize}
\techenv{Python, Fast API, Flask, Docker, Git, VS Code, Linux, GCP, Airflow}

\divider

\cvevent{Development Engineer}{Satelnet SpA.}{July 2019 -- January 2022}{Puerto Varas}{logos/satelnet_logo.png}
\begin{itemize}
  \item Designed and developed some Industrial Internet of Things (IIOT) solutions.
  \item Wrote automation scripts for data acquisition, storage and cloud syncing.
  \item Developed custom protocols for LoRa communications with ESP32 chips.
  \item \faStar Solved a Serial Communication challenge that saved a PoC.
\end{itemize}
\techenv{Python, Flask, VS Code, Linux, MongoDB, GCP, IIOT}

\divider

\cvevent{Cybersecurity Engineer}{NeoSecure S.A}{Sept. 2018 -- July 2019}{Providencia, Santiago}{logos/logo_neo.png}
\begin{itemize}
  \item Performed web and QRadar-based threat hunting.
  \item Participated in the Incident Response team activities.
  \item Wrote security newsletters and alerts.
  \item \faStar Wrote python integrations for Telegram and OTRS in Demisto.
\end{itemize}
\techenv{CrowdStrike, Demisto, SOAR, QRadar, Python, VS Code}

\divider

\cvevent{Tier 2 Security Analyst}{Neosecure S.A}{January 2018 -- Sept. 2018}{Providencia, Santiago}{logos/logo_neo.png}
\begin{itemize}
  \item Performed static malware analysis.
  \item Conducted ArcSight SIEM-based threat hunting.
  \item Automated tasks and reports for the analysts team.
\end{itemize}
\techenv{ArcSight, Python, Linux}

\divider

\cvevent{Tier 1 Security Analyst}{Neosecure S.A}{July 2016 -- December 2017}{Providencia, Santiago}{logos/logo_neo.png}
\begin{itemize}
  \item Monitored security and availability with ArcSight.
  \item Performed availability monitoring with Nagios.
  \item Delivered reports for clients.
  \item Automated tasks for Tier 1 analysts.
\end{itemize}
\techenv{ArcSight, Python, Linux, Nagios, VisualBasic}

\divider

\cvevent{Assistant Student: ICT Extension}{Universidad de Concepción}{March 2014 -- December 2015}{Concepción}{logos/marcaderecha.png}
\begin{itemize}
  \item Served as a Community Manager for career's social media.
  \item Supported career event organization.
  \item Created posts for the career's webpage.
\end{itemize}
\techenv{WordPress, HTML}

\divider

\cvevent{Assistant Student}{Universidad de Concepción}{March 2014 -- December 2015}{Concepción}{logos/marcaderecha.png}
\begin{itemize}
  \item Assisted in Applied Statistics and Random Processes courses.
  \item Worked as a Laboratory Assistant in the Data Networks Laboratory course.
  \item Led practical classes (solving practical guides).
  \item Developed practical guides and evaluations.
\end{itemize}
\techenv{Python, Linux, MATLAB, \LaTeX}

% \cvevent{Job Title 1}{Company 1}{Month 20XX -- Ongoing}{Location}
% \begin{itemize}
%   \item Job description 1
%   \item Job description 2
% \end{itemize}
% \divider

% \experience
%     {Start Date}{Job Title}{Company}{Location}
%     {End Date} 
%     {
%       \begin{itemize}
%         \item Job description 1
%         \item Job description 2
%         \item ...
%         \item Job description n
%       \end{itemize}
%     }{tech 1, tech 2, tech 3, ..., tech n}
%   \emptySeparator



  % \experience
  %   {January 2022}{Python Developer}{Tata Consultancy Services}{Santiago (Remote)}
  %   {March 2024} 
  %   {
  %     \begin{itemize}
  %       \item Contractor Developer in a financial company.
  %       \item Develop new features for the client's Tool.
  %       \item Improvements to the client's tool.
  %       \item Design, Schedule and Develop of the new products
  %       \item Support for internal users.
  %       \item Familiarity with Agile Methodologies, specifically Scrum.
  %     \end{itemize}
  %   }{Python, Fast API, Flask, Docker, Git, Visual Studio Code, Jenkins, Nexus, Linux, Google Cloud, Airflow}
  % \emptySeparator
  % \experience
  %   {July 2019} {Development Engineer}{Satelnet SpA.}{Puerto Varas}
  %   {January 2022}    
  %   {
  %     \begin{itemize}
  %       \item Develop of IIOT{\it (Industrial Intenet of Things)} Solutions.
  %       \item Develop automation scripts for data acquisition and storage.
  %       \item Design of software architecture and hardware for the solutions.
  %       \item Protocol development for LoRa based communications.
  %       \item Continuous improvements for the active services.                           
  %     \end{itemize}
  %   }{Python, Flask, Visual Studio Code, Linux, MongoDB, MariaDB, Google Cloud}
  % \emptySeparator
  % \experience
  %   {Sept. 2018}{Cybersecurity Engineer}{NeoSecure S.A}{Santiago}
  %   {July 2019}    
  %   {
  %     \begin{itemize}
  %       \item QRadar based Threat Hunting.
  %       \item Investigation of new threats and dynamic analysis in Sandbox.              
  %       \item Support for incident response.                   
  %       \item Writing of notices and security alerts.                          
  %       \item SOAR Demisto and EDR Crowdstrike Specialist.                      
  %       \item Development of company's software integrations for Demisto                                
  %     \end{itemize}
  %   }{CrowdStrike, Demisto, SOAR, QRadar, Python, Visual Studio Code}
  % \emptySeparator
  % \experience
  % {January 2018}{Tier 2 Security Analyst}{Neosecure S.A}{Santiago}
  % {Sept. 2018}{
  %   \begin{itemize}
  %     \item Static malware analysis.
  %     \item requirements attention related to investigation and analysis.
  %     \item ArcSight SIEM based Threat Hunting.
  %     \item Automation of some tasks and reports from Level 1 and 2 of Analysts.
  %   \end{itemize}
  % }{ArcSight, Python, Linux}
  % \emptySeparator
  % \experience
  % {July 2016}{Tier 1 Security Analyst}{Neosecure S.A}{Santiago}
  % {December 2017}{
  %   \begin{itemize}
  %     \item Security and availability monitoring with ArcSight.
  %     \item Availability monitoring based on Nagios.
  %     \item Alert checking in Linux platforms.
  %     \item Reports delivery for clients.
  %     \item Automation of some tasks related to Level 1 of Analysts.
  %   \end{itemize}
  % }{ArcSight, Python, Linux, Nagios, VisualBasic}
  % \emptySeparator
  % \experience
  % {March 2014}{Assistant Student: ICT Extension}{Universidad de Concepción}{Concepción}
  % {December 2015}{
  %   \begin{itemize}
  %     \item Community Manager of Career's social media.
  %     \item Support for career's events organization.
  %     \item Posts in the carrer's web page.
  %   \end{itemize}
  % }{WordPress, HTML}
  % \emptySeparator
  % \experience
  % {March 2014}{Assistant Student}{Universidad de Concepción}{Concepción}
  % {December 2015}{
  %   \begin{itemize}
  %     \item Assistant in Applied Statistics and Random Processes courses
  %     \item Laboratory Assistant in the course of Data Networks Laboratory.  
  %     \item Practical classes teacher (practical guides resolution).
  %     \item Development of practical guides and evaluations.
  %   \end{itemize}
  % }{Python, Linux, MATLAB, \LaTeX}


\switchcolumn

\cvsection{\faGraduationCap Educación}
\input{sections/formation.tex}
\vspace{.5em}
\cvsection{\faLanguage Idiomas}

\cvskill{Español}{5}
\cvskill{Inglés}{4}
\vspace{.5em}
\cvsection{\faKeyboard Habilidades}
\cvskill{Python, Bash}{4.5}
\cvskill{Docker, Git, terraform}{4}
\cvskill{SQL, NoSQL}{4}
\cvskill{Airflow, Jenkins}{4}
\cvskill{Agile}{4.5}
\cvskill{Software Engineering}{4}

\divider

Coding: \tagenv{C/C++, Java, JS}

\divider

Frameworks: \tagenv{\textbf{FastAPI}, Flask}

\divider

DB: \tagenv{MySQL, SQLite, BigQuery}

\divider

OS: \tagenv{\textbf{Linux}, macOS, Windows}

\divider

Tools: \tagenv{GCP, GitHub, Jenkins}

\divider

Mgmt: \tagenv{Agile, Scrum, Attlasian}\\

%% Supports X.5 values.

%% Yeah I didn't spend too much time making all the
%% spacing consistent... sorry. Use \smallskip, \medskip,
%% \bigskip, \vspace etc to make adjustments.

% \divider

\cvsection{\faFile Publicaciones}
\input{sections/publications.tex}

%\cvsection{Referees}

%\input{sections/references.tex}

\end{paracol}

\end{document}